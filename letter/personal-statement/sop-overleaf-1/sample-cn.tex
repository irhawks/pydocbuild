\documentclass{statementCN}

\Title{个人陈述}
\Name{任呈祥}
\Email{irhawks@hotmail.com}
\Web{}
\University{海南大学}

\begin{document}

\maketitle

留学个人陈述模板的开头部分

留学个人陈述在开头需要直奔主题,因为外国老师不喜欢绕来绕去的措辞,最好在开始就表现出积极的态度,以及表明自己为什么想要学这个专业,同时也不要忘记了阐述自己的独特之处。当然,因为这个是开始,所以要以简单明了为主。

译文:我是某某大学某某专业毕业的学生,想申请贵校的某某专业。我对某某专业很感兴趣,希望将来能成为这个行业的专家。我知道贵校的这个专业十分出名,在业界有很高的声誉,也培养了很多此行业的精英。我想成为此行列中的一员,所以我很想成为贵校的一份子,在该专业领域深造。

留学个人陈述模板的主体部分

需要开始阐述留学生的学术背景,如果有工作经验的话,需要描述一下工作背景如何和此专业相关

译文:为了在此专业深造,我做了非常广泛和彻底的准备。首先,我在xxx领域有非常扎实的背景。另外,在我本科学习期间,我加入到一个xxx俱乐部,专注于 xxx。我们项目处理一些xxx。我在校外xx大学的工作也使我得到了宝贵的实践机会。

当我在做本科水平工作的探索时,我发现我的兴趣和你们的项目非常相配。我的兴趣是xxx,特别令我着迷的是xxx课程。我沉迷于xxx方面的问题,这使得我 渴望成为这一领域的专家。虽然我了解到我现在的局限性,但是我相信我所有的热情,加上您学校对我的才能指导,将带领我通过严格的学习并且在两年硕士课程和 之后的博士课程将带领我进入到xxx领域。

留学个人陈述模板的结尾部分

申请人需要在留学个人陈述最后做一个预见性的未来规划,说明自己愿意为本行业做出的贡献和努力,以及自己想要达到的高度,再次确定自己将在这个领域走下去。

译文:从贵校毕业之后,我打算在一家xxx公司做xxx或者在学院里面做研究员。研究xxx的发展可以实现我的探索渴望。将来,我希望可以成为大学教授来分享我的知识和经验,用来指导新一代的xxx。

最后很重要的一点是,在写完陈述后,必须要检查是否有任何拼写或语法错误。一篇再好的文章,如果最基本的拼写或语法出现错误,都会直接给招生官留下不好的印象。

Good luck with writing!

\end{document}
