\documentclass[blue,casual,12pt]{ustcletter}
\usepackage{lipsum}

\address{安徽省合肥市金寨路96号}
\postcode{230026}
\tel{****-********}
\fax{****-********}
\mobile{(+*)*** **** ****}%only for formal theme
\email{email@ustc.edu.cn}
\homepage{//www.ustc.edu.cn/}

\begin{document}

USTC Letter是依照《中国科学技术大学视觉形象识别系统管理手册(试行)》的示例制作的LaTeX信纸模板,主要文件是ustcletter.cls以及对应的校名矢量图像。制作本模板的目的是方便TeX用户撰写带有科大标志的文档/信件,免去自行设置的繁琐过程,同时尽可能符合学校的相关规定,使生成的文件更加正式、美观。

模板提供三种形式的信纸布局——formal、casual、draft——以供用户根据不同需要选择。需要注意的是,draft模式暂未完成。同时提供了页面配色功能,提供三种配色方案——blue、red、black。

模板使用方式简介:
\begin{verbatim}
        \documentclass[<COLOR>,<THEME>,<OTHER>]{ustcletter}
\end{verbatim}


模板的实现简介:

模板基于article文档类定制,使用xeCJK提供中文支持,使用graphicx+tikz+calc宏包绘制信纸部件,使用color宏包实现配色调整,使用geometry+everypage+fancyhdr宏包控制页面输出。

本模板最佳匹配A4纸张——这个规格是按照学校规定严格设置,通过页面元素缩放来匹配其他类型纸张——这是通过等比缩放色带长度实现,色带宽度以及标志大小不变。

按照学校规定,配色只能使用有限的几种指定色彩,用户如果确实需要调整配色,可以自行修改配色方案。

注意,由于实现方式的原因,需要进行2次甚至更多次编译才能够输出最终页面。

模板的下载地址:
https://github.com/ywgATustcbbs?tab=repositories

~

~

\lipsum[1-5]

\end{document}}